
\chapter{Ferramentas} \label{chap:4}


\section{O Simulador Pipesim}
Sistemas de produção de petróleo são sistemas nos quais se deseja sempre produzir o máximo possível e de forma segura.
O PIPESIM, da Schlumberger, é um simulador de fluxo multifásico em regime permanente que pode ser utilizado tanto para o projeto como para planejamento de operações em campos de petróleo. Ele permite que sejam simuladas situações alternativas de forma mais rápida e segura que testes reais.

\section{Python}

O Python é uma linguagem de alto nível, interpretada, de desenvolvimento rápido muito utilizada para prototipação e na academia. Suas faciliades, como Jupyter Notebook, um ambiente que facilita a criação de documentos com códigos e análise de dados, e o Matplotlib, uma biblioteca que a torna quase tão poderosa quando o MatLab para visualização de dados, a tornaram muito utilizada também na academia.

A Linguagem surgiu em 1991, e segundo o toba [https://www.tiobe.com/tiobe-index/] é a quarta linguagem de programação mais popular atualmente.

Sua facilidade de uso, expêriencia prévia, e facilidades de visualização de dados foram decisivos para a sua escolha para este trabalho. 

\section{Interface Python e OpenLink}
Para interfaceamento do Pipesim com outros softwares, a Schlumberger disponibiliza uma API (Application Programming Interface) chamada OpenLink, idealizada para programação em C++, VBA, ou Visual Basic. Esta pode ser utilizada para interação programática com o Pipesim, habilitando a configuração de novos poços, alteração de poços existentes, análises, simulações, e automação de simulações. Com essa API, é possível variar os parametros do poço e avaliar as curvas características.

 Utilizando-se a biblioteca de Python pyWin32, é possível comunicar-se com a API, mas apenas em versões do Windows de 32 bits. Apesar de algumas peculiaridades no tratamento de arrays e outros tipos de dados, esta biblioteca permite o uso da API em uma linguagem de desenvolvimento mais rápido[citation needed] e com grandes facilidades de análise de dados [citation needed].

\section{NOMAD}
Para utilização do orthoMADS, foi escolhida a ferramenta NOMAD, uma implementação em C++ do orthoMADS desenvolvida pelo GERAD, um centro de pesquisa multi-universidades canadense [https://www.gerad.ca/en/about] para a solução de problemas de otimização sem-derivadas com problemas caixa-preta, aonde não se conhece o modelo explícito do problema. Bastando fornecer uma função objetivo e restrições, a ferramenta é capaz de encontrar um ponto ótimo para o problema. Ela também disponibiliza variações do algorítmo (como 2n ou n+1 bases) e a possibilidade de paralelismo (consultar mais que um ponto de forma concorrente).  

	

\section{Opal}
Para o uso do NOMAD com o Python, era sugerido, até o começo destes trabalhos, o uso da ferramenta Opal (A Framework for Optimization of Algorithms). Uma interface open-source de alto nível para a interface de Python com o NOMAD. Este framework dá a liberdade para configurar todos os parametros de otimização do NOMAD e também é capaz de paralelismo.

No entanto ele suporta apenas Python2.7 que está em desuso há mais de 10 anos [citation needed], de modo que foi necessário portá-lo para python3 [github], contribuição que foi enviada aos criadores originais [action needed].

É interessante ressaltar também que a partir da versão 3.8.0 do NOMAD, foi implementada um interface própria em Python (novamente Python2.7) que deve ser analisada em trabalhos futuros.



%%%%%%%%%%%%%%%
