\chapter{Conclusões e Perspectivas}
%
O principal objetivo deste trabalho, que foi utilizar métodos sem derivadas para sintonia de simuladores de escoamento multifásico de óleo e gás, foi alcançado.
%
Foi possível desenvolver e utilizar softwares capazes de interagir com simuladores de fluxo multifásico para, manipulando parâmetros, sintonizar modelos utilizando-se dados experimentais.
%

Desta forma foi mostrado que otimização sem derivadas é uma ferramenta que, dada uma função caixa-preta, é capaz de resolver problemas complexos, encontrando um ponto ótimo em tempo hábil sem necessidade da função explícita ou suas derivadas.
%

Para o futuro, é interessante explorar a opção do pyNOMAD, interface d NOMAD para Python nativa, diminuindo a complexidade envolvida no Opal, e consequentemente simplificando os trabalhos.
%

Outro caminho não explorado é a imposição de restrições adicionais, dependentes do valor da saída do simulador, que podem ser adicionados ao NOMAD para sintonia de redes mais complexas.
%

Além disso, é interessante analisar o Opal, de forma a verificar o motivo do seu uso aparentemente excessivo de poder computacional, além de fazer experimentos com mais núcleos.
%
