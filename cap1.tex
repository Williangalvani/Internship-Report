

\chapter{Introdução} \label{chap:1}



\section{Motivação}

Uma das dificuldades da produção de petróleo é controlar os processos de modo a produzir exatamente o necessário em cada estação. A produção em falta ou excesso de determinados componentes pode afetar os compoenentes do sistema, como bombas e separadores, as interfaces com sistemas auxiliares, como linhas de transporte, além de extressar os sistemas da plataforma.

  Para que campos de petróleo gerem o maior retorno financeiro possível, os processos devem idealmente estar sempre em um ponto ótimo de performance. Como estes são sistemas complexos e que variam com o tempo, simuladores são utilizados para estimar as respostas do sistema real. Para reproduzir as medidas observadas apropriadamente, estes simuladores devem ser rotineiramente re-sintonizados pelo engenheiro de processos. Esta sintonia pode ser uma tarefa cansativa e duradoura.
  
Um simulador bem sintonizado pode ser utilizado para modelagem matematica e desenvolvimento de  algoritmos de controle, assim como planejamento de operações.

Como a escala de produção destes sistemas costuma ser grande, até pequenas variações dos modelos reais podem causar um impacto financeiro considerável.


\section{Objetivos}

O objetivo final deste trabalho é o desenvolvimento de uma ferramenta capaz de auxiliar a otimização da produção em plataformas offshore. Nós pretendemos identificar, implementar, e testar métodos de otimização sem derivadas compatíveis para sintonizar um simulador de poços de petróleo.
Inicialmente os métodos a ser estudados são o Simplex de Nelder-Mead, e OrthoMADs (implementação NOMAD).
É esperado que o método desenvolvido ajude a aliviar a carga sobre o engenheiro do processo e melhorar a produção, de modo a aliviar uma de suas tarefas.

 

\section{Estrutura}

Este relatório está dividido em quatro seções principais.
Na seção 2, é dada uma visão geral dos campos de petróleo e como eles funcionam, explicando a estrutura de uma FPSO, assim como cada um de seus componentes. A medida que os sistema são descritos, as variáveis usadas para os testes serão destacadas.

Na seção 3, damos uma apresentação geral dos conceitos de otimização sem derivada, demonstrando cada um dos métodos utilizados.

Na seção 4, nós apresentamos a estrutura da aplicação, como os componentes se conectam, as configuração e condições dos testes, entradas e saídas, e discussão dos resultados.

A seção 5 contém as conclusões dos experimentos.

 

%%%%%%%%%%%%%%%%%%%%%%%