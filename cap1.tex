

\chapter{Introduction} \label{chap:1}



\section{Motivation}

In order to get the maximum financial returns from an oilfield, the processes must be always at a point of maximum performance. Since the oilfield systems are complex and change with time, simulators are used to approximate the real world system. In order to reproduce the observed measurements properly, 
the simulators are tuned routinely by the process engineer. This tuning is often a cumbersome and time-consuming task.

A well tuned simulator can be user for process modeling and prediction by control and optimization algorithms.

How long it takes to tune the simulator reflects on how long the production can be optimized, thus potentially affecting the oilfields revenue.


\begin{itemize}

 \item \textcolor{blue}{Oil well simulators are tuned routinely in order to reproduce observed measurements, a process that can be cumbersome and time-consuming}.
 
 \item \textcolor{blue}{The simulator of oil wells should be adequately tuned before it can be used for process modeling and prediction by control and optimization algorithms.}
 
 \item \textcolor{blue}{The degree of accuracy can have major production and economic impact.}
\end{itemize}


\section{Goals}

O objetivo final deste trabalho é o desenvolvimento de uma ferramenta capaz de auxiliar a otimização da produção em plataformas offshore. Nós pretendemos identificar, implementar, e testar métodos de otimização sem derivadas compatíveis para sintonizar um simulador.
Inicialmente os métodos a ser estudados são O Simplex de Nelder-Mead, Baricentro, e OrthoMADs (implementação NOMAD).
É esperado que o método desenvolvido ajude a aliviar a carga sobre o engenheiro, mudando sua função para monitoração do sistema.

\begin{itemize}
 \item \textcolor{blue}{Study and identify suitable derivative-free optimization methods for an oil well simulator}.
 
 \item \textcolor{blue}{Implement and test derivative-free optimization algorithms for an oilfield simulator.}
 
\end{itemize}


\section{Report Structure}

Este relatório está dividido em quatro seções principais.
Na seção 2, é dada uma visão geral dos campos de petróleo e como eles funcionam, explicando a estrutura de uma FPSO, assim como cada um de seus componentes. A medida que os sistema são descritos, as variáveis usadas para BATATA e DOCE DE LEITE serão destacadas.

Na seção 3, damos uma apresentação geral dos conceitos de otimização sem derivada, demonstrando cada um dos métodos utilizados.

Na seção 4, nós apresentamos a estrutura da aplicação, como os componentes se conectam, as configuração e condições dos testes, entradas e saídas, e discussão dos resultados.

A seção 5 contém as conclusões dos experimentos.

 

%%%%%%%%%%%%%%%%%%%%%%%