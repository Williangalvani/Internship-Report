

\chapter{Introdução} \label{chap:1}



\section{Motivação}
Na indústria de petróleo e gás, frequentemente duas ou mais plataformas de produção conectam-se à redes comuns para o escoamento da produção. 
%
Este escoamento precisa ser planejado para que as restrições físicas de fluxo e pressão sejam respeitadas.
%
Para que estes valores estejam corretos, é necessário controlar os processos de modo a produzir exatamente o necessário em cada plataforma.
%
Além disso, para que campos de petróleo gerem o maior retorno financeiro possível, os processos devem idealmente estar sempre em um ponto ótimo de performance.
%
A produção em falta ou excesso de determinados componentes pode afetar os elementos do sistema, como bombas e separadores, ou as interfaces com sistemas auxiliares, como linhas de transporte.
% 
Outros problemas ocasionados pelo excesso de produção são o estresse dos sistemas da plataforma e perdas desnecessárias com desgaste ou subprodução.

Como estes são sistemas complexos e que variam com o tempo, simuladores são utilizados para estimar as respostas do sistema real à diversas condições. 
%
Para reproduzir as medidas observadas apropriadamente, estes simuladores devem ser rotineiramente re-sintonizados pelo engenheiro de processos. 
%
Esta sintonia pode ser uma tarefa cansativa e duradoura.
  
Um simulador bem sintonizado pode ser utilizado para modelagem matemática de novos modelos, desenvolvimento e testes de algoritmos de controle, planejamento de operações e análise de riscos, entre outros.	

Como a escala de produção destes sistemas costuma ser grande, até pequenas variações dos modelos reais podem causar impactos consideráveis na produção.


\section{Objetivos}

O objetivo final deste trabalho é o desenvolvimento de uma ferramenta capaz de auxiliar a sintonia de simuladores de fluxo multifásico de petróleo e gás utilizando dados de plantas reais em plataformas em alto mar. 
%
Pretende-se identificar, implementar e testar métodos de otimização sem derivadas, os quais sejam compatíveis para sintonizar tais simuladores.
%
É esperado que a ferramenta seja capaz de aproximar em pouco tempo as curvas de produção de um simulador aos dados coletados em campo.
%
Inicialmente os métodos a serem estudados são o Simplex de Nelder-Mead \cite{Singer:2009}, com uma implementação própria, o OrthoMADs \cite{DBLP:journals/siamjo/AbramsonADD09}, e OrthoMADS com SGTELIB (estes dois últimos com a implementação NOMAD \cite{Nomad}).
%
É esperado que o método desenvolvido ajude desenvolver ferramentas de automação para aliviar a carga sobre o engenheiro de processos, e melhorar a produção.
%

%
\section{Estrutura}

Este relatório está dividido em cinco capítulos.
%
No capítulo 1, é dada uma introdução geral ao relatório.
%
O capítulo 2 traz uma visão geral dos campos de petróleo e como eles funcionam, explicando a estrutura de uma FPSO\footnote{Floating Production, Offloading and Storage, um tipo de estrutura utilizada para extração de petróleo e gás.}, e seus componentes principais. 
%
A medida que os sistema são descritos, as variáveis usadas para os testes serão destacadas.
%
O capítulo 3 oferece uma apresentação geral dos conceitos de otimização sem derivada, e são mostrados cada um dos métodos utilizados.
%
No capítulo 4, são apresentadas as ferramentas envolvidas no trabalho.
%
No capítulo 5 é apresentado o problema, a estrutura da solução proposta, como os componentes se conectam, as configurações e as condições dos testes, e variáveis envolvidas.
%
Os capítulos 6 e 7 contêm os resultados dos experimentos realizados.
%
Considerações finais e perspectivas para trabalhos futuros são apresentados no capítulo 8.

 

%%%%%%%%%%%%%%%%%%%%%%%