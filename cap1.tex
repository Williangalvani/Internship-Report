

\chapter{Introduction} \label{chap:1}



\section{Motivation}

In order to get the maximum financial returns from an oilfield, the processes must be always at a point of maximum performance. Since the oilfield systems are complex and change with time, simulators are used to approximate the real world system. In order to reproduce the observed measurements properly, 
the simulators are tuned routinely by the process engineer. This tuning is often a cumbersome and time-consuming task.

A well tuned simulator can be user for process modeling and prediction by control and optimization algorithms.

How long it takes to tune the simulator reflects on how long the production can be optimized, thus potentially affecting the oilfields revenue.


\begin{itemize}

 \item \textcolor{blue}{Oil well simulators are tuned routinely in order to reproduce observed measurements, a process that can be cumbersome and time-consuming}.
 
 \item \textcolor{blue}{The simulator of oil wells should be adequately tuned before it can be used for process modeling and prediction by control and optimization algorithms.}
 
 \item \textcolor{blue}{The degree of accuracy can have major production and economic impact.}
\end{itemize}


\section{Goals}

The ultimate goal is to optimize the production and profit. We aim to identify, implement, and test suitable derivative-free optimization methods to tune the Marlim simulator. The methods being studied are the Simplex, Baricenter, and OrthoMADS aswell as NOMAD's implementation.
We hope the system helps relieving the engineer's burden of tuning the simulator, moving him to a overseeing task. 

\begin{itemize}
 \item \textcolor{blue}{Study and identify suitable derivative-free optimization methods for an oil well simulator}.
 
 \item \textcolor{blue}{Implement and test derivative-free optimization algorithms for an oilfield simulator.}
 
\end{itemize}


\section{Report Structure}

This report is organized in four main sections. On section 2, an overview of the oilfield systems and how they work is given, explaining the structure of an Floating Production, Storage and Offloading (FPSO) unit, as well as each of their main components. As the systems are presented, the variables used for control or feedback on next sections will be highlighted.

On section 3, we give a general presentation of the Derivative-Free optimization concept, showcasing the utilized methods. 

On section 4, we present the application structure, how everything connects, the experiments setup and conditions, input and output data, and discuss the results.

Section 5 contains the experiments conclusions.

 



%%%%%%%%%%%%%%%%%%%%%%%