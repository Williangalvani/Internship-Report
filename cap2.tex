

\chapter{Modelagem e Simulação de Poços de Petróleo e Gás} \label{chap:2}


\section{Produção em Alto Mar}

A produção de petróleo e gás pode ser estruturada em diversos layouts diferentes.
% 
Em águas rasas, até 100m, é comum a utilização de Complexos de Águas Rasas, os quais são compostos de diversas plataformas, geralmente conectadas por pontes, cada uma com uma função principal, como acomodações, riser, e geração de energia.
%
Já entre 100 e 500m, é possível utilizar-se de bases de gravidade, grandes estruturas cônicas e ocas de concreto, que tem função tanto estrutural quanto de armazenamento. 
%

Torres articuladas são uma estrutura semelhante às bases de gravidade, mas as base é conectada a plataforma por uma torre articulada, de forma a absorver os impactos de correntes marinhas e ventos.
%

FPSOs (Floating, Production, Storage and Offloading) são geralmente construídas a partir de navios-tanque, com uma torre giratória na proa ou central aonde são conectadas as tubulações, permitindo que o navio se alinhe livremente ao vento e correntes marítimas. 
%
As FPSOs dominam os novos campos de petróleo, já que os novos campos de exploração estão predominantemente em áreas de maior profundidade, e são o foco nesse trabalho.

\section{Floating Production, Offloading, Storage}
Uma FPSO(Floating Production, Storage and Offloading) é uma estrutura utilizada para produção de óleo e gás em grandes profundidades. Trata-se geralmente de um cargueiro modificado, conectado a poços no fundo por risers e umbilicais, e ancorado no fundo do mar.

Estas estruturas são capazes de produzir, estocar, e descarregar óleo de forma autônoma. O gás no entanto costuma ser re-utilizado para a injeção na forma de gas-lift, uma forma artificial de aumentar a produção de poços, tranportado por tubulações, ou liquefeito para o transporte.

No fundo do oceano podem existir um ou mais poços, que são conectados a um manifold centralizando a coleta, que por sua vez conecta-se à FPSO por meio de flowlines e risers.

O gás, por ter sua armazenagem e transporte mais complicados, frequentemente é utilizado para gas-lift, ou escoado por tubulações submarinas, mas estas tubulações frequentemente tem restrições de fluxo, requerendo certas especificações de composição e pressão, de forma que mais processamento é necessário.

Pequenas quantidades de hidrocarbonetos, em situações especiais, como desestabilização temporária de algum sistema, podem ser queimadas no flare. Embora fosse tradicional o uso continuo do flare no passado, atualmente, por pressões ambientais, eles são utilizados apenas esporadicamente.

\begin{itemize}

\item \textcolor{blue}{Discuss in the general terms the structure of an offshore oilfield, which consists of subsea oil wells with risers connecting to a production platform.}


\item \textcolor{red}{Discuss in general terms the surface processing facilities, valves, separator, compressor, flare and exportation.}

\end{itemize}



\section{Poço de Petróleo}

um poço de petróleo é geralmente uma perfuração na superfície terrestre conectando a um reservatório subterrâneo. 
%
Primeiramente, é iniciada uma perfuração na superfície utilizando-se brocas. 
%
Durante a perfuração, periodicamente o furo precisa ser reforçado, para evitar que todo o poço colapse. 
%
Este reforço é composto por cilindros estruturais que são instalados nas paredes externas da perfuração, que podem ser cimentados no exterior. 
%
O poço então é finalizado, etapa em que são abertos buracos no envoltório estrutural para a passagem de óleo e gás. 
%
Os métodos para criar estes orifícios variam de buracos pré-existentes nos envoltórios a explosões controladas.
%


Após a conclusão dos orifícios, é bombeado líquido de fraturação para abrir novos canais no reservatório e facilitar o escoamento.
%
O poço então recebe a arvore de natal, composta por um choke, indicador de pressão, e outras valulas, e é conectado a um manifold.
%
Em plataformas submarinas, o poço principal conecta diretamente ao manifold, enquanto poços satélites se conectam por linhas de fluxo multifásico.
%
O manifold então conecta-se ao riser, uma tubulação responsável por transportar os fluidos até a superfície.
%


O riser, dependendo do tipo de óleo e condições climáticas, pode incluir aquecimento ou diluentes para facilitar o transporte.
%


Ao chegar a superfície, a mistura vai para um separador, geralmente uma grande estrutura cilíndrica aonde são separadas as diferentes fases do fluxo (gás, óleo e água), basicamente por ação da gravidade e diferença de densidade.
%


A separação pode ocorrer em diversos estágios, com diferentes pressões, para separar diferentes componentes.
%
A partir deste estágio, o óleo pode ser armazenado e enviado para refinarias por tubulações, ou armazenado no casco e descarregado por navios tanque semanalmente.
%
Já o gás (ó o gááás) não costuma ser armazenado, e tende a ser re-utilizado ou enviado diretamente por tubulações, após um pre-processamento para se adequar as condições necessárias pela tubulação.
%
 

\section{Gas Lift}

A medida que um reservatório é esvaziado, a pressão interna dele abaixa, o que em alguns casos faz com que não haja pressão o suficiente para manter a vazão desejada.
%
Uma solução para este problema é o gas lift, um método para a elevação artificial de fluidos, largamente empregado na indústria do petróleo.
%
Este método consiste na injeção de gás pressurizado nos poços facilitando o deslocamento dos fluidos até a plataforma de produção. 
%
Ele funciona tanto pela efeito da energia da expansão do gás injetado, quanto pela diminuição da densidade média do fluido no riser, causada pela mistura de mais gás à mistura que é produzida.
%
Uma característica do processo de gas lift visível neste trabalho é que existe um ponto ótimo de injeção, pois após um certo ponto, adicionar mais gás apenas aumenta a produção do próprio gás, ao invés de produzir mais líquidos.
%
Para a utilização do gas lift são necessários compressores para pressurizar o gás novamente e reinjetá-lo no poço.





\begin{itemize}

\item \textcolor{blue}{Give a brief presentation about Marlim and/or Pipesim.}

\item \textcolor{blue}{Present some curves of oil wells modeled in Pipesim.}


\end{itemize}
