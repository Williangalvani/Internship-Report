

\chapter{Oil Well Modeling and Simulation} \label{chap:2}


\section{Offshore Oil Production}

A produção de petróleo e gás pode ser estruturada em diversos layouts diferentes. Em águas razas, até 100m, é comum a utilização de Complexos de Aguas Razas, os quais são compostos de diversas plataformas, geralmente conectadas por pontes, cada uma com uma função principal, como acomodações, riser, e geração de energia.
Já entre 100 e 500m, é possível utilizar-se de bases de gravidade, grandes estruturas conicas e ocas de concreto, que tem função tanto estrutural quanto de armazenamento. Torres articuladas são uma estrutura semelhante às bases de gravidade, mas as base é conectada a plataforma por uma torre articulada, de forma a absorver os impactos de correntes marinhas e ventos.
FPSOs (Floating, Production, Storage and Offloading) são geralmente construídas a partir de navios-tanque, com uma torre giratória na proa ou central aonde são conectadas as tubulações, permitindo que o navio se alinhe livremente ao vento e correntes maritmas. As FPSOs dominam os novos campos de petróleo, já que os novos campos de exploração estão predominantemente em áreas de maior profundidade, e são o foco nesse trabalho.

\subsection{Estrutura de uma FPSO}
Uma FPSO(Floating Production, Storage and Offloading) é uma estrutura utilizada para produção de óleo e gás em grandes profundidades. Trata-se geralmente um cargueiro modificado, conectado a poços no fundo por risers e umbilicais, e ancorado no fundo do mar.

Estas estruturas são capazes de produzir, estocar, e descarregar óleo de forma autônoma. O gás no entando costuma ser re-utilizado para a injeção na forma de gas-lift, uma forma artificial de aumentar a produção de poços.

No fundo do oceano podem existir um ou mais poços, que são conectados a um manifold, que por sua vez conectam-se à FPSO.

O gás, por ter sua armazenagem e transporte mais complicados, frequentemente é utilizado para gas-lift, ou escoado por tubulações submarinas, mas estas tubulações frequentemente tem restrições de fluxo, e na falta de para onde escoá-lo, ele pode queimado no flare [citation needed].

\subsection{Gas Lift}
 O Gás Lift é um método para a elevação artificial de
fluidos, largamente empregado na indústria do petróleo. Ele método consiste na
injeção de gás nos poços facilitando o deslocamento dos fluidos até a plataforma de produção. Ele funciona tanto pela energia da expansão do gás injetado, quanto pela diminuição da densidade média do flúido no riser, causada pela mistura de mais gás à mistura.


\begin{itemize}

\item \textcolor{blue}{Discuss in the general terms the structure of an offshore oilfield, which consists of subsea oil wells with risers connecting to a production platform.}


\item \textcolor{blue}{Discuss in general terms the surface processing facilities, valves, separator, compressor, flare and exportation.}

\end{itemize}



\section{Poço de Petróleo}

um poço de petróleo é geralmente uma perfuração na superfície terrestre conectando a um reservatório subterrâneo. 

Primeiramente, um furo é feito na superfície utilizando-se brocas. A seguir, são instalados cilindros estruturais nas paredes externas deste buraco, que podem ser cimentadas no exterior. 

O poço então é finalizado, etapa em que são abertos buracos no envoltório estrutural para a passagem de óleo e gás. Os métodos para criar estes orifícios variam de buracos pré-existentes nos envoltórios a explosões controladas.

Após a conclusão dos orifícios, é bombeado líquido de fraturação para abrir novos canais no reservatório e facilitar o escoamento.



\begin{itemize}

\item \textcolor{blue}{Give a brief presentation about Marlim and/or Pipesim.}

\item \textcolor{blue}{Present some curves of oil wells modeled in Pipesim.}


\end{itemize}
