

\chapter{Oil Well Modeling and Simulation} \label{chap:2}


\section{Offshore Oil Production}

A produção em si pode ser estruturada em diversos layouts diferentes. Em águas razas, até 100m, é comum a utilização de Complexos de Aguas Razas, os quais são compostos de diversas plataformas, geralmente conectadas por pontes, cada uma com uma função principal, como acomodações, riser, e geração de energia.
Já entre 100 e 500m, é possível utilizar-se de bases de gravidade, grandes estruturas conicas e ocas de concreto, que tem função tanto estrutural quanto de armazenamento. Torres articuladas são uma estrutura semelhante às bases de gravidade, mas as base é conectada a plataforma por uma torre articulada, de forma a absorver os impactos de correntes marinhas e ventos.
FPSOs (Floating, Production, Storage and Offloading) são geralmente construídas a partir de navios-tanque, com uma torre giratória na proa ou central aonde são conectadas as tubulações, permitindo que o navio se alinhe livremente ao vento e correntes maritmas. As FPSOs dominam os novos campos de petróleo, já que os novos campos de exploração estão predominantemente em áreas de maior profundidade.

Nas próximas seções são descritos os principais componentes do sistema de produção em uma FPSO.

\subsubsection{Wellbore}

O buraco em si. Cavado na rocha, o poço propicia acesso direto ao recurso buscado, seja ele petróleo ou gás. A estrutura do poço em si pode variar bastante dependendo da localização, profundidade, composição da rocha e tipo de recurso procurado.
 \\

\subsubsection{Wellhead}
A cabeça do poço é o componente mais upstream, fazendo a interface entre o poço em si e os sistemas de extração. Ela pode conter vários instrumentos e conexões para testes, e é comumente chamada de árvore de natal.

\subsubsection{Choke}
It's used to manipulate the flow out of the wellhead.


\subsubsection{Manifold}



\subsubsection{Riser}



\subsubsection{Flow Lines}

\subsubsection{...}

\begin{itemize}

\item \textcolor{blue}{Discuss in the general terms the structure of an offshore oilfield, which consists of subsea oil wells with risers connecting to a production platform.}

\item \textcolor{blue}{Discuss in general terms the surface processing facilities, valves, separator, compressor, flare and exportation.}

\end{itemize}


\section{Satellite Oil Wells}

\textcolor{blue}{Discuss the structure of offshore oil wells that operate with continuous lift-gas injection (gas-lift).}



\section{Oil Well}


\begin{itemize}

\item \textcolor{blue}{Give a brief presentation about Marlim and/or Pipesim.}

\item \textcolor{blue}{Present some curves of oil wells modeled in Pipesim.}


\end{itemize}
